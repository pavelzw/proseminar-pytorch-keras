\section{Introduction | Einleitung}
Mandatory. Questions like: What is the topic of this work, what's the broader context (topic of the proseminar), why is it relevant?

\section{PyTorch}
\subsection{Daten laden und präparieren}

\lstinputlisting[label=py:load-data-pytorch, language=Python, caption=MNIST Datensatz mit TorchVision herunterladen.]{code/load-data-pytorch.py}

\subsection{Modellerstellung}

\lstinputlisting[label=py:load-data-pytorch, language=Python, caption=Modellerstellung in PyTorch.]{code/create-model-pytorch.py}

\subsection{Modell trainieren}

\lstinputlisting[label=py:load-data-pytorch, language=Python, caption=Modell trainieren in PyTorch.]{code/train-model-pytorch.py}

\section{TensorFlow und Keras}
\subsection{Daten laden und präparieren}

\lstinputlisting[label=py:load-data-pytorch, language=Python, caption=MNIST Datensatz mit \texttt{keras.datasets} herunterladen.]{code/load-data-tensorflow.py}

\subsection{Modellerstellung}

\lstinputlisting[label=py:load-data-pytorch, language=Python, caption=Modellerstellung in TensorFlow.]{code/create-model-tensorflow.py}

\subsection{Modell trainieren}

\lstinputlisting[label=py:load-data-pytorch, language=Python, caption=Modell trainieren in TensorFlow.]{code/train-model-tensorflow.py}

\subsection{CNN Implementierung}

\lstinputlisting[label=py:load-data-pytorch, language=Python, caption=CNN Implementierung in Keras.]{code/create-cnn-keras.py}

\lstinputlisting[label=py:load-data-pytorch, language=Python, caption=Transfer Learning in Keras.]{code/transfer-learning-keras.py}

\lstinputlisting[label=py:load-data-pytorch, language=Python, caption=Funktionale Modelle in Keras.]{code/transfer-learning-functional-keras.py}

\section{Zusammenfassung und Fazit}
Mandatory. Short summary of the most important aspects of the report.
If possible: What are open challenges?

\newpage
\section{\LaTeX Examples}
As a help to get started with this template. To be deleted for submission.
\subsection{Citation examples}
\citet{campbell:2017} define the stages of information processing in a nervous system as: "sensory input, integration, and motor output". \\
The stages of information processing in a nervous system are defined as: "sensory input, integration, and motor output" \citep{campbell:2017}. 

\subsection{Table example}
\begin{table}[htbp]
    \centering
    \begin{tabular}{lrl}
    \toprule
    labels   & numbers & annotation \\
    \midrule
    abra     &  1.23   & \textbf{this is important}\\
    cadabra  &  2.34   & this isn't\\
    \bottomrule
    \end{tabular}
    \caption{Some random numbers}
    \label{tab:random}
\end{table}

\subsection{Figure examples}
This is a png file, it gets blurry when you zoom in:
\begin{figure}[htbp]
    \centering
    \includegraphics[width=.7\textwidth]{figures/leaky_integration.png}
    \caption{Symbolic representation of a leaky integrating neuron.}
    \label{fig:leaky_integration}
\end{figure}

This is an eps file, it is always sharp:\\
Notice how the formatting option "[htbp]" allows for the figure to be moved around to page \pageref{fig:activation_function}. Hence, it is best to rather write: The eps file in figure \ref{fig:activation_function} always stays sharp.
\begin{figure}[htbp]
    \centering
    \includegraphics[width=.7\textwidth]{figures/activation_functions}
    \caption{Shapes of a parametrized tanh activation function.}
    \label{fig:activation_function}
\end{figure}

\subsection{Math example}
The state update of the leaky integrating neuron in figure \ref{fig:leaky_integration} can be formulated as:
\begin{align}
    x_i(t+1) &= \lambda_i \cdot \left(W_{i,j} \cdot U_j(t)\right) + (1-\lambda_i) \cdot \theta_i(t)
    \label{eq:leaky_integration}
\end{align}

\subsection{Footnote example}
The implementation is available on github\footnote{https://github.com/schniewmatz/recurrence}.