\section*{Abstract | Zusammenfassung}
Über die Jahre hinweg gab es bereits viele verschiedene Frameworks für das Programmieren 
mit Algorithmen im Bereich maschinelles Lernen. Von diesen vielen Frameworks 
haben sich schlussendlich zwei Frameworks durchgesetzt, nämlich PyTorch und TensorFlow. 
In diesem Proseminar betrachten wir diese beiden Frameworks und gehen darauf ein, 
wie man mit diesen Multi Layer Perceptrons und Convolutional Neural Networks 
erstellen und trainieren kann und vergleichen die beiden Frameworks schlussendlich.
PyTorch, TensorFlow und Keras sind hervorragende Tools, mit denen man mit 
relativ wenig Aufwand neuronale Netze erstellen kann. Wenn man vorhat, sich 
mit Machine Learning zu beschäftigen, sollte man auf jeden Fall mit mindestens einem der 
Frameworks gut umgehen können.